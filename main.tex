\documentclass[a4paper,10pt,reqno]{amsart}

\usepackage[utf8]{inputenc}
\usepackage[foot]{amsaddr}
\usepackage{amsmath,amsfonts,amssymb,amsthm,mathrsfs,bm}
\usepackage[margin=0.95in]{geometry}
\usepackage{color}
\usepackage[dvipsnames]{xcolor}
\usepackage{cancel}
\usepackage{empheq}
\usepackage{float}



\usepackage{etoolbox}

% Modifications to amsart ToC-related macros...
\makeatletter
\let\old@tocline\@tocline
\let\section@tocline\@tocline
% Insert a dotted ToC-line for \subsection and \subsubsection only
\newcommand{\subsection@dotsep}{4.5}
\newcommand{\subsubsection@dotsep}{4.5}
\patchcmd{\@tocline}
  {\hfil}
  {\nobreak
     \leaders\hbox{$\m@th
        \mkern \subsection@dotsep mu\hbox{.}\mkern \subsection@dotsep mu$}\hfill
     \nobreak}{}{}
\let\subsection@tocline\@tocline
\let\@tocline\old@tocline

\patchcmd{\@tocline}
  {\hfil}
  {\nobreak
     \leaders\hbox{$\m@th
        \mkern \subsubsection@dotsep mu\hbox{.}\mkern \subsubsection@dotsep mu$}\hfill
     \nobreak}{}{}
\let\subsubsection@tocline\@tocline
\let\@tocline\old@tocline

\let\old@l@subsection\l@subsection
\let\old@l@subsubsection\l@subsubsection

\def\@tocwriteb#1#2#3{%
  \begingroup
    \@xp\def\csname #2@tocline\endcsname##1##2##3##4##5##6{%
      \ifnum##1>\c@tocdepth
      \else \sbox\z@{##5\let\indentlabel\@tochangmeasure##6}\fi}%
    \csname l@#2\endcsname{#1{\csname#2name\endcsname}{\@secnumber}{}}%
  \endgroup
  \addcontentsline{toc}{#2}%
    {\protect#1{\csname#2name\endcsname}{\@secnumber}{#3}}}%

% Handle section-specific indentation and number width of ToC-related entries
\newlength{\@tocsectionindent}
\newlength{\@tocsubsectionindent}
\newlength{\@tocsubsubsectionindent}
\newlength{\@tocsectionnumwidth}
\newlength{\@tocsubsectionnumwidth}
\newlength{\@tocsubsubsectionnumwidth}
\newcommand{\settocsectionnumwidth}[1]{\setlength{\@tocsectionnumwidth}{#1}}
\newcommand{\settocsubsectionnumwidth}[1]{\setlength{\@tocsubsectionnumwidth}{#1}}
\newcommand{\settocsubsubsectionnumwidth}[1]{\setlength{\@tocsubsubsectionnumwidth}{#1}}
\newcommand{\settocsectionindent}[1]{\setlength{\@tocsectionindent}{#1}}
\newcommand{\settocsubsectionindent}[1]{\setlength{\@tocsubsectionindent}{#1}}
\newcommand{\settocsubsubsectionindent}[1]{\setlength{\@tocsubsubsectionindent}{#1}}

% Handle section-specific formatting and vertical skip of ToC-related entries
% \@tocline{<level>}{<vspace>}{<indent>}{<numberwidth>}{<extra>}{<text>}{<pagenum>}
\renewcommand{\l@section}{\section@tocline{1}{\@tocsectionvskip}{\@tocsectionindent}{}{\@tocsectionformat}}%
\renewcommand{\l@subsection}{\subsection@tocline{1}{\@tocsubsectionvskip}{\@tocsubsectionindent}{}{\@tocsubsectionformat}}%
\renewcommand{\l@subsubsection}{\subsubsection@tocline{1}{\@tocsubsubsectionvskip}{\@tocsubsubsectionindent}{}{\@tocsubsubsectionformat}}%
\newcommand{\@tocsectionformat}{}
\newcommand{\@tocsubsectionformat}{}
\newcommand{\@tocsubsubsectionformat}{}
\expandafter\def\csname toc@1format\endcsname{\@tocsectionformat}
\expandafter\def\csname toc@2format\endcsname{\@tocsubsectionformat}
\expandafter\def\csname toc@3format\endcsname{\@tocsubsubsectionformat}
\newcommand{\settocsectionformat}[1]{\renewcommand{\@tocsectionformat}{#1}}
\newcommand{\settocsubsectionformat}[1]{\renewcommand{\@tocsubsectionformat}{#1}}
\newcommand{\settocsubsubsectionformat}[1]{\renewcommand{\@tocsubsubsectionformat}{#1}}
\newlength{\@tocsectionvskip}
\newcommand{\settocsectionvskip}[1]{\setlength{\@tocsectionvskip}{#1}}
\newlength{\@tocsubsectionvskip}
\newcommand{\settocsubsectionvskip}[1]{\setlength{\@tocsubsectionvskip}{#1}}
\newlength{\@tocsubsubsectionvskip}
\newcommand{\settocsubsubsectionvskip}[1]{\setlength{\@tocsubsubsectionvskip}{#1}}

% Adjust section-specific ToC-related macros to have a fixed-width numbering framework
\patchcmd{\tocsection}{\indentlabel}{\makebox[\@tocsectionnumwidth][l]}{}{}
\patchcmd{\tocsubsection}{\indentlabel}{\makebox[\@tocsubsectionnumwidth][l]}{}{}
\patchcmd{\tocsubsubsection}{\indentlabel}{\makebox[\@tocsubsubsectionnumwidth][l]}{}{}

% Allow for section-specific page numbering format of ToC-related entries
\newcommand{\@sectypepnumformat}{}
\renewcommand{\contentsline}[1]{%
  \expandafter\let\expandafter\@sectypepnumformat\csname @toc#1pnumformat\endcsname%
  \csname l@#1\endcsname}
\newcommand{\@tocsectionpnumformat}{}
\newcommand{\@tocsubsectionpnumformat}{}
\newcommand{\@tocsubsubsectionpnumformat}{}
\newcommand{\setsectionpnumformat}[1]{\renewcommand{\@tocsectionpnumformat}{#1}}
\newcommand{\setsubsectionpnumformat}[1]{\renewcommand{\@tocsubsectionpnumformat}{#1}}
\newcommand{\setsubsubsectionpnumformat}[1]{\renewcommand{\@tocsubsubsectionpnumformat}{#1}}
\renewcommand{\@tocpagenum}[1]{%
  \hfill {\mdseries\@sectypepnumformat #1}}

% Small correction to Appendix, since it's still a \section which should be handled differently
\let\oldappendix\appendix
\renewcommand{\appendix}{%
  \leavevmode\oldappendix%
  \addtocontents{toc}{%
    \protect\settowidth{\protect\@tocsectionnumwidth}{\protect\@tocsectionformat\sectionname\space}%
    \protect\addtolength{\protect\@tocsectionnumwidth}{2em}}%
}
\makeatother

% #1 (default is as required)

% #2

% #3
\makeatletter
\settocsectionnumwidth{2em}
\settocsubsectionnumwidth{2.5em}
\settocsubsubsectionnumwidth{3em}
\settocsectionindent{1pc}%
\settocsubsectionindent{\dimexpr\@tocsectionindent+\@tocsectionnumwidth}%
\settocsubsubsectionindent{\dimexpr\@tocsubsectionindent+\@tocsubsectionnumwidth}%
\makeatother

% #4 & #5
\settocsectionvskip{10pt}
\settocsubsectionvskip{0pt}
\settocsubsubsectionvskip{0pt}

% #6 & #7
% See #3

% #8
\renewcommand{\contentsnamefont}{\bfseries\Large}

% #9
\settocsectionformat{\bfseries}
\settocsubsectionformat{\mdseries}
\settocsubsubsectionformat{\mdseries}
\setsectionpnumformat{\bfseries}
\setsubsectionpnumformat{\mdseries}
\setsubsubsectionpnumformat{\mdseries}

% #10
% Insert the following command inside your text where you want the ToC to have a page break
\newcommand{\tocpagebreak}{\leavevmode\addtocontents{toc}{\protect\clearpage}}

% #11
\let\oldtableofcontents\tableofcontents
\renewcommand{\tableofcontents}{%
  \vspace*{-\linespacing}% Default gap to top of CONTENTS is \linespacing.
  \oldtableofcontents}

\usepackage{mathtools,enumerate,mathrsfs,graphicx}
\usepackage{epstopdf}
\usepackage{hyperref}
\usepackage[dvipsnames]{xcolor}

\usepackage{latexsym}


\definecolor{CommentGreen}{rgb}{0.0,0.4,0.0}
\definecolor{Background}{rgb}{0.9,1.0,0.85}
\definecolor{lrow}{rgb}{0.914,0.918,0.922}
\definecolor{drow}{rgb}{0.725,0.745,0.769}
\definecolor{grey}{RGB}{240,240,240}

\usepackage{listings}
\usepackage{textcomp}
\lstloadlanguages{Matlab}%
\lstset{
    language=Matlab,
    upquote=true, frame=single,
    basicstyle=\small\ttfamily,
    backgroundcolor=\color{Background},
    keywordstyle=[1]\color{blue}\bfseries,
    keywordstyle=[2]\color{purple},
    keywordstyle=[3]\color{black}\bfseries,
    identifierstyle=,
    commentstyle=\usefont{T1}{pcr}{m}{sl}\color{CommentGreen}\small,
    stringstyle=\color{purple},
    showstringspaces=false, tabsize=5,
    morekeywords={properties,methods,classdef},
    morekeywords=[2]{handle},
    morecomment=[l][\color{blue}]{...},
    numbers=none, firstnumber=1,
    numberstyle=\tiny\color{blue},
    stepnumber=1, xleftmargin=10pt, xrightmargin=10pt
}

\numberwithin{equation}{section}
\synctex=1

\hypersetup{
    unicode=false, pdftoolbar=true, 
    pdfmenubar=true, pdffitwindow=false, pdfstartview={FitH}, 
    pdftitle={ELE2024 Coursework}, pdfauthor={A. Author},
    pdfsubject={ELE2024 coursework}, pdfcreator={A. Author},
    pdfproducer={ELE2024}, pdfnewwindow=true,
    colorlinks=true, linkcolor=red,
    citecolor=blue, filecolor=magenta, urlcolor=cyan
}


% CUSTOM COMMANDS
\renewcommand{\Re}{\mathbf{re}}
\renewcommand{\Im}{\mathbf{im}}
\newcommand{\R}{\mathbb{R}}
\newcommand{\N}{\mathbb{N}}
\newcommand{\C}{\mathbb{C}}
\newcommand{\lap}{\mathscr{L}}
\newcommand{\dd}{\mathrm{d}}
\newcommand{\smallmat}[1]{\left[ \begin{smallmatrix}#1 \end{smallmatrix} \right]}

%opening
\title[ELE2024 Coursework]{ELE2024 Control Coursework}

\author[J. Boden]{John Boden}
\author[J. Tutty]{Jonathan Tutty}
\author[S. Paine]{Simon Paine}

\address[J. Boden, J. Tutty and S. Paine]{Email addresses: \href{mailto:jboden03@qub.ac.uk}{jboden03@qub.ac.uk},
\href{mailto:jtutty01@qub.ac.uk}{jtutty01@qub.ac.uk} and
\href{mailto:spaine01@qub.ac.uk}{spaine01@qub.ac.uk}.}
\thanks{Link to Git Hub code:.... 
        Version 0.0.1. Last updated:~\today.}
\begin{document}

\maketitle

\section{Part A}

NOTE: CTRL + F9 is the shortcut to compile on Papeeria
% * <jboden03@qub.ac.uk> 12:28:30 04 Dec 2020 UTC+0000:
% This is useful
% ^ <Jonathan Tutty> 15:11:46 05 Dec 2020 UTC+0000:
% Good to know, ty

\subsection{Problem A1}\label{sec:a1}
% * <jboden03@qub.ac.uk> 14:10:20 04 Dec 2020 UTC+0000:
% NB. I have collapsed Part A1 - it is here
The objective of this question was to use the information provided in the problem description to derive a system of ordinary differential equation describing how the input voltage, V, affects the position, $x$, of the ball on the inclined plane.
\newline An inerital frame of reference was introduced where counterclockwise rotations were taken to be positive as shown in Figure \ref{fig:system}.

\begin{figure}[h]
\centering
\includegraphics[width=10cm]{figures/courseworktemplate3}
\caption{The system of a wooden ball on an inclined plane, involving a spring, damper and electromagnet controlled by a voltage V. Resolved forces are indicated in green.}
\label{fig:system}
\end{figure}

\newline Initally, we can derive an equation to represent the static friction, $T$, acting on the system as follows:
\newline It is known that the friction creates a torque on the ball with respect to its centre of mass. From Figure xyz, we can see this is equal to the Friction multiplied by the radius, i.e. $Tr$. 
\newline Applying Newton's law for rotational motion leads to the following equation:

\begin{align}
\label{1.1}
Tr &= I\ddot{\theta}
\end{align}

\newline where \ddot{\theta} represents the ball's angular acceleration. Recalling that the angular acceleration and linear acceleration, \ddot{x}, are linked by the following equation:

\begin{align}
\ddot{x} &= \ddot{\theta}r \label{1.2} \\
\intertext{Rearranging, we see that:}
\ddot{\theta} &= \frac{\ddot{x}}{r} \label{1.3}
\end{align}

\newline Substitution of \ref{1.3} into \ref{1.1} leads to the following equation for the friction:

\begin{align}
Tr &= I\bigg(\frac{\ddot{x}}{r}\bigg)\nonumber\\ 
Tr &= \frac{I_n\ddot{x}}{r} \nonumber \\ 
\intertext{therefore:}
T &= \frac{I_n\ddot{x}}{r^2} \label{1.4}
\end{align}

\newline We can also substitute for $I_n$, the moment of inertia of the ball. 
\newline For a ball where an axis runs through its centre, $I_n = \frac{2}{5}mr^2$.
\newline The equation for friction therefore becomes:

\begin{align}
T &= \frac{\frac{2}{5}m\cancel{r^2}\ddot{x}}{\cancel{r^2}} \nonumber \\
T &= \frac{2}{5}m\ddot{x} \label{1.5}
\end{align}
\vspace{3mm}

\newline A system of ordinary differential equations that describe the system can now be derived as follows:
\begin{align}
F_{mag} + mg\sin(\phi) - b\dot{x} - k(x - d) - T &= m\ddot{x} \nonumber \\
c \frac{I^2}{y^2} + mg\sin(\phi) - b\dot{x} - k(x - d) - T &= m\ddot{x} \nonumber \\
\intertext{substituting for T and also for y:}
c \frac{I^2}{(\delta - x)^2} + mg\sin(\phi) - b\dot{x} - k(x - d) - \frac{2}{5}m\ddot{x} &= m\ddot{x} \nonumber \\
c \frac{I^2}{(\delta - x)^2} + mg\sin(\phi) - b\dot{x} - k(x - d) &= m\ddot{x} + \frac{2}{5}m\ddot{x}\nonumber \\
c \frac{I^2}{(\delta - x)^2} + mg\sin(\phi) - b\dot{x} - k(x - d) &= \frac{7}{5}m\ddot{x}\nonumber
\end{align}

rearranging to determine $\ddot{x}$:

\begin{align}
\ddot{x} &= \frac{c \frac{I^2}{(\delta - x)^2} + mg\sin(\phi) - b\dot{x} - k(x-d)}{\frac{7}{5}m} \label{1.6}
\end{align}

rearranging this leads to an equation summarising the balls acceleration:

\begin{align}
\ddot{x} &= \frac{5}{7m}\bigg(c \frac{I^2}{(\delta - x)^2} + mg\sin(\phi) - b\dot{x} - k(x-d)\bigg) \label{1.7}
\end{align}

\newline From the circuit given in the problem ddescription, we can establish that:

\begin{align}
V = IR + L\dot{I} \nonumber \\
\intertext{therefore:}
\dot{I} = \frac{V - IR}{L} \label{1.8}
\end{align}

\newline From the problem description, we are told that:
\begin{align}
L = L_0 + L_1\exp(-\alpha y) \label{1.9}
\end{align}

\newline Substituting \ref{1.9} into \ref{1.8} leads to the following equation:

\begin{align}
\dot{I} = \frac{V - IR}{L_0 + L_1\exp(-\alpha y)} \label{1.10}
\end{align}

\newline In order to have the overal equations in terms of the variables $x$ and $V$ only, $y$ is replaced with the equivalent term `$\delta - x$':

\begin{align}
\dot{I} = \frac{V - IR}{L_0 + L_1\exp(-\alpha (\delta - x))} \label{1.11}
\end{align}

\newline Overall, we can describe the system using equations \ref{1.7} and \ref{1.11}

\subsection{Problem A2}\label{sec:a2}
We can now write the system in state space representation, using the two equations obtained in \ref{sec:a1}.

We can firstly define four new variables as follows:

\begin{align}
x_1 &= x \label{1.12} \\
x_2 &= \dot{x_1} (= \dot{x}) \label{1.13} \\
x_3 &= I \label{1.14}
\end{align}

\newline We can now rewrite \ref{1.7} and \ref{1.10} using these new variables as follows:

\begin{align}
\dot{x_2} &= \frac{5}{7m}\bigg(c \frac{{x_3}^2}{(\delta - x_1)^2} + mg\sin(\phi) - bx_2 - k(x_1-d)\bigg) \label{1.15} \\
\dot{x_3} &= \frac{V - x_3 R}{L_0 + L_1\exp(-\alpha y)} \label{1.16}
\end{align}

\newline Equations \ref{1.15} and \ref{1.16} can be used to express the system in a state space representation as follows:

\begin{align}
\begin{bmatrix}
\dot{x}
\end{bmatrix} 
= 
\begin{bmatrix}
\dot{x_1}\\
\dot{x_2}\\
\dot{x_3}
\end{bmatrix}
= 
\begin{bmatrix}
x_2 \\
\frac{5}{7m}\bigg(c \frac{{x_3}^2}{(\delta - x_1)^2} + mg\sin(\phi) - bx_2 - k(x_1-d)\bigg)\\
\frac{V - x_3 R}{L_0 + L_1\exp(-\alpha y)}
\end{bmatrix}
\label{1.17}
\end{align}

\newline The states of the system are $x_1$ and $x_2$, while the input is $x_3$, the current. All other terms in the equation are constants.



\subsection{Problem A3}\label{sec:a3}
The equlibrium points of the system are characterised in \ref{eq:eq_points}.

\begin{subequations}\label{eq:eq_points}
\begin{align}
      x^{eq}_{2} &= 0 \label{1.18a} \\
     \frac{5}{7m}\bigg(c \frac{({x_3}^{eq})^2}{(\delta - x_1^{eq})^2} + mg\sin(\phi) - bx_2^{eq} - k(x_1^{eq}-d)\bigg) &= 0 \label{1.18b} \\
     \frac{V - x_3^{eq}R}{L_0 + L_1\exp(-\alpha(\delta - x_1^{eq}))} &= 0 \label{1.18c}
\end{align}
\end{subequations}




\subsection{Problem A4}\label{sec:a4}
We subtract (\ref{1.13}) and (\ref{eq:eq_points}) by parts to obtain

\begin{subequations}
     \begin{align}
          \dot{x_1} &= x_2 - x_2^{eq} \\
          \dot{x_2} &=  \frac{5}{7m}\bigg(c \frac{{x_3}^2}{(\delta - x_1)^2} + mg\sin(\phi) - bx_2 - k(x_1-d)\bigg) - \frac{5}{7m}\left(c \frac{({x_3}^{eq})^2}{(\delta - x_1^{eq})^2} + mg\sin(\phi) - bx_2^{eq} - k(x_1^{eq}-d)\right) \label{eq:x2dot}\\
          \dot{x_3} &= \frac{V - x_3R}{L_0 + L_1\exp(-\alpha(\delta - x_1))} - \frac{V - x_3^{eq}R}{L_0 + L_1\exp(-\alpha(\delta - x_1^{eq}))} \label{eq:x3dot}
     \end{align}
\end{subequations}\newline
We define the functions

\begin{subequations}
    \begin{align}
         \phi(x_1, x_2, x_3) = \dot{x_2} &= \frac{5}{7m}\left[\frac{cx_3^2}{(\delta - x_2)^2} - \frac{c(x_3^{eq})^2}{(\delta - x_2^{eq})^2} - k(x_1 - x_1^{eq}) - b(x_2 - x_2^{eq})\right] \\
         \psi(V, x_1, x_3) = \dot{x_3} &= \frac{V - x_3R}{L_0 + L_1\exp(-\alpha(\delta - x_1))} - \frac{V^{eq} - x_3^{eq}R}{L_0 + L_1\exp(-\alpha(\delta - x_1^{eq}))}
    \end{align}
\end{subequations}

We linearised $\phi$ at an equlibrium $(x_1^{eq}, x_2^{eq}, x_3^{eq})$. Its partial derivatives with respect to $x_1, x_2$ and $x_3$ at the equlibrium point are

\begin{subequations}\label{eq:phiDeriv}
     \begin{align}
          \frac{\partial\phi}{\partial x_1}\Bigg|_{x_1^{eq}, x_2^{eq}, x_3^{eq}} &= \frac{5}{7 m}\left(\frac{2 c (x_3^{eq})^{2}}{\left(\delta - x_{1}^{eq}\right)^{3}} - k\right) \\
          \frac{\partial\phi}{\partial x_2}\Bigg|_{x_1^{eq}, x_2^{eq}, x_3^{eq}} &= - \frac{5 b}{7 m} \\
          \frac{\partial\phi}{\partial x_3}\Bigg|_{x_1^{eq}, x_2^{eq}, x_3^{eq}} &= \frac{10 c x_{3}^{eq}}{7 m \left(\delta - x_{1}^{eq}\right)^{2}}
     \end{align}
\end{subequations}

therefore, $\phi(x_1, x_2, x_3)$ can be approximated by
\begin{equation}
     \phi(x_1, x_2, x_3) \approx \frac{5}{7 m}\left(\frac{2 c (x_{3}^{eq})^{2}}{\left(\delta - x_{1}^{eq}\right)^{3}} - k\right)(x_1 - x_1^{eq}) - \frac{5 b}{7 m}(x_2 - x_2^{eq}) + \frac{10 c x_{3}^{eq}}{7 m \left(\delta - x_{1}^{eq}\right)^{2}}(x_3 - x_3^{eq})
\end{equation}

We linearised $\psi$ at an equlibrium $(V^{eq}, x_1^{eq}, x_3^{eq})$. Its partial derivatives with respect to $V, x_1$ and $x_3$ at the equlibrium point are

\begin{subequations}\label{eq:psiDeriv}
     \begin{align}
        \frac{\partial\psi}{\partial V}\Bigg|_{V^{eq}, x_1^{eq}, x_3^{eq}} &= \frac{1}{L_{0} + L_{1} e^{- \alpha \left(\delta - x_{1}^{eq}\right)}} \\
          \frac{\partial\psi}{\partial x_2}\Bigg|_{V^{eq}, x_1^{eq}, x_3^{eq}} &= - \frac{L_{1} \alpha \left(- R x_{3}^{eq} + V\right) e^{- \alpha \left(\delta - x_{1}^{eq}\right)}}{\left(L_{0} + L_{1} e^{- \alpha \left(\delta - x_{1}^{eq}\right)}\right)^{2}} \\
          \frac{\partial\psi}{\partial x_3}\Bigg|_{V^{eq}, x_1^{eq}, x_3^{eq}} &= - \frac{R}{L_{0} + L_{1} e^{- \alpha \left(\delta - x_{1}^{eq}\right)}}
     \end{align}
\end{subequations}

therefore, $\psi(V, x_1, x_3)$ can be approximated by
\begin{equation}
     \psi(V, x_1, x_3) \approx \frac{1}{L_{0} + L_{1} e^{- \alpha \left(\delta - x_{1}^{eq}\right)}}(V - V^{eq}) - \frac{L_{1} \alpha \left(- R x_{3}^{eq} + V\right) e^{- \alpha \left(\delta - x_{1}^{eq}\right)}}{\left(L_{0} + L_{1} e^{- \alpha \left(\delta - x_{1}^{eq}\right)}\right)^{2}}(x_1 - x_1^{eq}) - \frac{R}{L_{0} + L_{1} e^{- \alpha \left(\delta - x_{1}^{eq}\right)}}(x_3 - x_3^{eq})
\end{equation}

N.B.: The code used to determine all the partial derivatives can be found at XYZ

We now introduce the deviation variables $\overline{x}_1 = x_1-x_1^{eq}, \overline{x}_2 = x_2-x_2^{eq}, \overline{x}_3 = x_3-x_3^{eq}$ and $\overline{V} = V-V^{eq}$. The linearised system then becomes

\begin{subequations}\label{eq:linear_system}
     \begin{align}
          \dot{\overline{x}}_1 &= \overline{x}_2 \\
          \dot{\overline{x}}_2 &= \frac{5}{7 m}\left(\frac{2 c (x_{3}^{eq})^{2}}{\left(\delta - x_{1}^{eq}\right)^{3}} - k\right)\overline{x}_1  - \frac{5 b}{7 m}\overline{x}_2 + \frac{10 c x_{3}^{eq}}{7 m \left(\delta - x_{1}^{eq}\right)^{2}}\overline{x}_3\\
          \dot{\overline{x}}_3 &= \frac{1}{L_{0} + L_{1} e^{- \alpha \left(\delta - x_{1}^{eq}\right)}}\overline{V} - \frac{L_{1} \alpha \left(- R x_{3}^{eq} + V\right) e^{- \alpha \left(\delta - x_{1}^{eq}\right)}}{\left(L_{0} + L_{1} e^{- \alpha \left(\delta - x_{1}^{eq}\right)}\right)^{2}}\overline{x}_1 - \frac{R}{L_{0} + L_{1} e^{- \alpha \left(\delta - x_{1}^{eq}\right)}}\overline{x}_3
     \end{align}
\end{subequations}

\subsection{Problem A5}\label{sec:a5}\hfill

\setlength{\jot}{5pt} % Increases spacing between equations

\begin{figure}[H]
% * <Jonathan Tutty> 16:26:09 07 Dec 2020 UTC+0000:
% I also added the "float" package so you can use "H" to force latex to put an image where you want it.
     \includegraphics[width = 0.5\textwidth]{figures/tf_block.eps}
     \caption{Block diagram of the system's tranfer function}
     \label{fig:tfBlock}
\end{figure}

The equation for the transfer function of a dynamical system is

\begin{equation}
     G(s) = \frac{X(s)}{U(s)} \label{eq:TF} \\
\end{equation}

\begin{center}
     where $X(s)$ is the input (s-domain), \\
     and $U(s)$ is the output (s-domain)
\end{center}
\vspace{10pt}

The transfer function of the system can be represented by the block diagram shown in figure \ref{fig:tfBlock}. In order to determine $G(s)$, some constants are defined for the partial derivatives calculated in equations \ref{eq:phiDeriv} and \ref{eq:psiDeriv} in order to simplify the linearised system expressed in equations \ref{eq:linear_system}

\begin{subequations} \label{eq:SysLT}
    \begin{align}
         A_1 &= \frac{5}{7 m}\left(\frac{2 c (x_{3}^{eq})^{2}}{\left(\delta - x_{1}^{eq}\right)^{3}} - k\right) \label{eq:A1} \\
         A_2 &= -\frac{5 b}{7 m} \label{eq:A2} \\
         A_3 &= \frac{10 c x_{3}^{eq}}{7 m \left(\delta - x_{1}^{eq}\right)^{2}} \label{eq:A3} \\
         \intertext{and}
         B_1 &= \frac{1}{L_{0} + L_{1} e^{- \alpha \left(\delta - x_{1}^{eq}\right)}} \label{eq:B1} \\
         B_2 &= - \frac{L_{1} \alpha \left(- R x_{3}^{eq} + V\right) e^{- \alpha \left(\delta - x_{1}^{eq}\right)}}{\left(L_{0} + L_{1} e^{- \alpha \left(\delta - x_{1}^{eq}\right)}\right)^{2}} \label{eq:B2} \\
         B_3 &= - \frac{R}{L_{0} + L_{1} e^{- \alpha \left(\delta - x_{1}^{eq}\right)}} \label{eq:B3}
    \end{align}
\end{subequations}
\vspace{1pt}

Then the Laplace transform is applied to equations \ref{eq:linear_system}. This gives

\begin{subequations} \label{eq:SysLT}
    \begin{align}
         s\overline{X}_1(s) &= \overline{X}_2(s) \label{eq:sX1} \\
         s\overline{X}_2(s) &= A_1(s)\overline{X}_1(s) + A_2(s)\overline{X}_2(s) + A_3(s)\overline{X}_3(s) \label{eq:sX2} \\
         s\overline{X}_3(s) &= B_1(s)\overline{V}(s) + B_2(s)\overline{X}_1(s) + B_3(s)\overline{X}_3(s) \label{eq:sX4}
    \end{align}
\end{subequations}
\vspace{1pt}

In order to find the tranfer function, an expression for $\overline{X}_1(s)$ in terms of $\overline{V}(s)$ must be found. To do this two expressions for $\overline{X}_3(s)$ can be found and then solved simultaneously.\\
The first expression is determined by substituting \ref{eq:sX1} into \ref{eq:sX2}

\begin{align*}
     s^2\overline{X}_1(s) &= A_1(s)\overline{X}_1(s) + A_2(s)\overline{X}_2(s) + A_3(s)\overline{X}_3(s) \\
     \iff s^2\overline{X}_1(s) &= A_1(s)\overline{X}_1(s) + sA_2(s)\overline{X}_1(s) + A_3(s)\overline{X}_3(s) \\
     \iff A_3(s)\overline{X}_3(s) &= s^2\overline{X}_1(s) - A_1(s)\overline{X}_1(s) - sA_2(s)\overline{X}_1(s)
\end{align*}
\vspace{1pt}

Therefore,

\begin{equation}
     \overline{X}_3(s) =\frac{s^2\overline{X}_1(s) - A_1(s)\overline{X}_1(s) - sA_2(s)\overline{X}_1(s)}{A_3(s)} \label{eq:X3_1}
\end{equation}
\vspace{1pt}

Likewise, the second expression is determined by rearranging \ref{eq:sX4}:

\begin{align*}
     s\overline{X}_3(s) &= B_1(s)\overline{V}(s) + B_2(s)\overline{X}_1(s) + B_3(s)\overline{X}_3(s) \\
     \iff s\overline{X}_3(s) - B_3(s)\overline{X}_3(s) &= B_1(s)\overline{V}(s) + B_2(s)\overline{X}_1(s) \\
     \iff \overline{X}_3(s)\left(s - B_3(s)\right) &= B_1(s)\overline{V}(s) + B_2(s)\overline{X}_1(s)
\end{align*}
\vspace{1pt}

Therefore,

\begin{equation}
     \overline{X}_3(s) &= \frac{B_1(s)\overline{V}(s) + B_2(s)\overline{X}_1(s)}{s - B_3(s)} \label{eq:X3_2}
\end{equation}
\vspace{1pt}

Equations \ref{eq:X3_1} and \ref{eq:X3_2} can now be solved simultaneously

\begin{align*}
     \frac{s^2\overline{X}_1(s) - A_1(s)\overline{X}_1(s) - sA_2(s)\overline{X}_1(s)}{A_3(s)} \label{eq:X3_1} &= \frac{B_1(s)\overline{V}(s) + B_2(s)\overline{X}_1(s)}{s - B_3(s)} \label{eq:X3_2} \\
     \iff \left(s^2\overline{X}_1(s) - A_1(s)\overline{X}_1(s) - sA_2(s)\overline{X}_1(s)\right) \left(s - B_3(s)\right) &= A_3(s)B_1(s)\overline{V}(s) + A_3(s)B_2(s)\overline{X}_1(s) \\
     \iff \overline{X}_1(s)\left(s^2 - A_1(s) - sA_2(s)\right) \left(s - B_3(s)\right) - A_3(s)B_2(s)\overline{X}_1(s) &= A_3(s)B_1(s)\overline{V}(s) \\
     \iff \overline{X}_1(s)\left[\left(s^2 - A_1(s) - sA_2(s)\right) \left(s - B_3(s)\right) - A_3(s)B_2(s)\right] &= A_3(s)B_1(s)\overline{V}(s)
\end{align*}
\vspace{1pt}

Therefore,

\begin{equation}
     \overline{X}_1(s) = \frac{A_3(s)B_1(s)\overline{V}(s)}{\bigl(s^2 - sA_2(s) - A_1(s)\bigr) \bigl(s - B_3(s)\bigr) - A_3(s)B_2(s)} \label{eq:X1}
\end{equation}
\vspace{1pt}

Using equation \ref{eq:TF}, and given that equation \ref{eq:X1} is the input and $\overline{V}(s)$ is the output, it can be seen that the transfer function of the system is

\begin{empheq}[box={\setlength{\fboxsep}{10pt}\colorbox{grey}}]{equation}\label{eq:SysTF}
% * <Jonathan Tutty> 15:45:52 07 Dec 2020 UTC+0000:
% I added the "empheq" package to emphasise the most important equations as I think it makes the report a bit easier to read.
% ^ <jboden03@qub.ac.uk> 15:21:05 29 Dec 2020 UTC+0000:
% smart thinking
         G(s) = \frac{\overline{X}_1(s)}{\overline{V}(s)} = \frac{A_3(s)B_1(s)}{\bigl(s^2 - sA_2(s) - A_1(s)\bigr) \bigl(s - B_3(s)\bigr) - A_3(s)B_2(s)}
\end{empheq}

\section{Part B}

\subsection{Problem B1}\label{sec:b1}
Include all info on start of B1 here. The following relates to the second half of B1:
We also need to determine the equilibrium voltage and current as a function of $x^e$.

Using (\ref{1.18b}) and applying (\ref{1.18a}), we can establish that
\begin{align}
\textcolor{orange}{c \frac{({x_3}^{eq})^2}{(\delta - x_1^{eq})^2}} + \textcolor{blue}{mg\sin(\phi) - \cancelto{0}{bx_2^{eq}} - k(x_1^{eq}-d)} &= 0 \label{PartBstart}
\end{align}
Let us consider first the blue section of (\ref{PartBstart}). For the ball to equilibriate, there must be forces acting upon it in both directions. When the ball is at $x_{min}$, we know that the orange part of the equation will be acting downwards. The combination of the other 2 non-zero components must not become less than 0, otherwise the ball will not be able to settle in an equilibium position.
We can therefore say the following,
\begin{align}
mg \sin (\phi) - k(x_{min} - d) &= 0 \nonumber \\
\Leftrightarrow k(x_{min} - d) &= mg \sin (\phi) \nonumber \\
\Leftrightarrow x_{min} - d &= \frac {mg \sin (\phi) }{k } \nonumber \\
\Leftrightarrow x_{min} &= \frac {mg \sin (\phi) }{k } + d \label{PartB1xmin}
\end{align}

Let us now consider the orange section of (\ref{PartBstart}). For the ball to equilibriate, the ball cannot go beyond the centre of the inductor, otherwise both the overall forces from the orange and blue section acting on the system would be to the left, and hence the ball could not equilibriate in this area. In order to find the maximum $x$-position of the ball, we can find when this orange section equals 0 and determine the corresponding $x$ value.
\begin{align}
c \frac{({x_3}^{eq})^2}{(\delta - x_1^{eq})^2} &= 0 \label{x1_max}
\end{align}
The only mathematically possible way this can occur is if
\begin{align}
c(x_3^{eq})^2 = 0 \label{eqn: B1x_3eqn}
\end{align}


Using (\ref{1.18c}), we can determine the following:
\begin{align}
V - x_3^{eq}R &= 0 \nonumber \\
\Leftrightarrow x_3^{eq} &= \frac{V}{R} \label{eqn: x3eq}
\end{align}
where $x_3^{eq}$ is the equilibrium current.
\end{align}
Application of (\ref{eqn: x3eq})to (\ref{eqn: B1x_3eqn}) leads to the following:
\begin{align}
c(\frac{V}{R})^2 = 0 \label{orangeeqn}
\end{align}
It is clear that the only variable in (\ref{orangeeqn}) is $V$, therefore, for this equation to equal 0, $V$ must equal 0.
In order to determine the corresponding $x_{max}$ value when $V = 0$, we can use (\ref{1.18b}) again to determine an equation in $x_1$ and $V$.

From (\ref{1.18b}), which is rewritten below for clarity, we can determine the following:

\begin{align}
\frac{5}{7m}\bigg(c \frac{({x_3}^{eq})^2}{(\delta - x_1^{eq})^2} + mg\sin(\phi) - bx_2^{eq} - k(x_1^{eq}-d)\bigg) &= 0 \nonumber \\
\intertext{Either $\frac{5}{7m} = 0$ or the bracted section of the equation equals 0. As it is impossible for $\frac{5}{7m} = 0$ given the constant value of $m$, this leads to the following conclusion:}
c \frac{({x_3}^{eq})^2}{(\delta - x_1^{eq})^2} + mg\sin(\phi) - bx_2^{eq} - k(x_1^{eq}-d) &= 0 \nonumber \\
c(x_3^{eq})^2 + mg\sin(\phi)(\delta - x_1^{eq})^2 - k(x_1^{eq} - d)(\delta - x_1^{eq})^2 &= 0 \nonumber \\
\intertext{Substituting \ref{eqn: x3eq}}
c\frac{V^2}{R^2} + mg\sin(\phi)(\delta^2 - 2 \delta x_1^{eq} + (x_1^{eq})^2) - k(x_1^{eq} - d)(\delta^2 - 2 \delta x_1^{eq} + (x_1^{eq})^2) &= 0 \nonumber \\
c\frac{V^2}{R^2} + mg\sin(\phi)(\delta^2 - 2 \delta x_1^{eq} + (x_1^{eq})^2) - k(\delta^2 x_1^{eq} - 2\delta (x_1^{eq})^2 + (x_1^{eq})^3 -d\delta^2 + 2d\delta x_1^{eq} - d(x_1^{eq})^2) &= 0 \nonumber \\
c\frac{V^2}{R^2} + (x_1^{eq})^3(-k) + (x_1^{eq})^2(mg\sin(\phi) + 2k\delta +dk) + x_1^{eq}(-2\delta mg\sin(\phi) - k\delta^2 -2dk\delta) + (\delta^2mg\sin(\phi) + kd\delta^2) &= 0 \nonumber \\
\bigg(\frac{R^2}{c}\bigg((x_1^{eq})^3(k) + (x_1^{eq})^2(-mg\sin(\phi) - 2k\delta -dk) + x_1^{eq}(2\delta mg\sin(\phi) + k\delta^2 +2dk\delta) \nonumber \\  + (-\delta^2mg\sin(\phi) - kd\delta^2)\bigg)\bigg)^\frac{1}{2} &= V \nonumber \\ 
\intertext{Substituing for the constants,}
\Leftrightarrow 774896.55((x_1^{eq})^3) - 1333971.97((x_1^{eq})^2) + 751982.01x_1^{eq} - 137991.22 &= V^2 \label{eqn: B1deriv}
\end{align}

A plot of this equation is shown below.
\begin{figure}[h]
\centering
\includegraphics[width=10cm]{figures/ProblemB1}
\caption{A plot to determine the maximum equilibrium voltage, and the corresponding $x$-position....}
\label{fig:B1plot}
\end{figure}

From the plot, we can see that the value of $x_{max}$ we were trying to determine when $V = 0$ is found at 0.65m. This is equal to $\delta$, therefore satisfying Problem B1. It can also be shown that the second point at which this plot meets the x-axis is at $x_1 = 0.42148$m. This is the same value obtained following substitution of the constants into (\ref{PartB1xmin}), thereby verifying our derivation of this point as well.
Overall, we can agree with Problem B1 and state that
\begin{align}
x_{min} < x^e < x_{max},
\end{align}
where $x_{min} = d + \frac{mg\sin(\phi)}{k}$ and $x_{max} = \delta$.

From the plot, we can also see that the maximum equilibrium voltage value is $37.0111$ V. The corresponding $x$-position where this maximum voltage occurs is $0.4976$ m.


We now need to linearise the system at the equilibrium point that corresponds to $x=0.75x_{min}+0.25x_{max}$.
This can be done by substituting the equations for $x_{min}$, $x_{max}$ into this equation along with any constants provided in the problem description as follows.

\begin{align}
x &= 0.75x_{min} + 0.25 x_{max}\nonumber \\
x_1^{eq} &= 0.75 \bigg(d + \frac{mg \sin{\phi}}{k}\bigg) + 0.25 \delta \nonumber \\
\intertext{therefore, substituting for the constant values:}
x_1^{eq} &= 0.75 \bigg(0.42 + \frac{(0.425)(9.81) \sin{(42 \circ)}}{1880}\bigg) + 0.25(0.65) \nonumber \\
x_1^{eq} &= 0.47861 \label{eqn: xvalue}
\end{align}

We can also determine the corresponding equilibrium voltage by substitution of $x_1^{eq}$ into (\ref{eqn: B1deriv}).
This leads to the following:
\begin{align}
(V^e)^2 &= 1300.24005 \nonumber \\
\Leftrightarrow V^e &= \pm 36.0588 V \nonumber \\
\intertext{We will only consider the Voltage being a positive value, therefore}
\Leftrightarrow V^e &= 36.0588 V \label{eqn: Ve value}
\end{align}



\subsection{Problem B2}\label{sec:b2}


\subsection{Problem B3}\label{sec:b3}


\subsection{Problem B4}\label{sec:b4}




\subsection{Problem B5}\label{sec:b5}

When designing a controller for the system that manipulates the voltage to steer the $x$-position of the ball towards a set point $x^{sp}$, it should meet some desired characteristics; the ball should move to the set point quickly and without oscillation, and it should not exhibit constant bias. To prove that the controller is good, a plot should be provided of the trajectory of $x(t)$ that clearly shows the aforementioned characteristics. For example, if it is specified that after 0.5~s, $x$ should not deviate more than 0.1~cm from $x^{sp}$, the graph should demonstrate this.

\section{Part C}

\subsection{Probelm C1}\label{sec:c1}
The following function is defined in the Problem Description:
\begin{align}
f(t) = \ln t^3, t>0 \label{eqn: C1function}
\end{align}
The general form of the Laplace Transform is as follows,
\begin{align}
F(s) = (\mathscr{L}f)(s) =  \int_{0}^{\infty} e^{-st}f(t) \,dt  \label{eqn: Laplace}
\end{align}
Application of this to our function, $f(t)$ leads to the following:
\begin{align}
F(s) &= \int_{0}^{\infty} e^{-st} \ln t^3 \,dt, s>0  \nolabel \\
\Leftrightarrow F(s) &= \int_{0}^{\infty} e^{-st} 3\ln t \,dt  \label{eqn: C1beginning}
\end{align}
Integrating (\ref{eqn: C1function}) by parts will not simplify the integral - rather it will lead to the contrary. Instead a substitution method can be applied as follows.
\begin{align}
\intertext{let}
u &=st \label{eqn: C1u} \\
\Leftrightarrow \frac{\mathrm{d}u}{\mathrm{d}t} &= s \nolabel \\
\Leftrightarrow \mathrm{d}t &= \frac{1}{s} \mathrm{d}u \label{eqn: c1dt}
\end{align}
The limits of the integral also need to be considered:
\begin{align*}
\intertext{from (\ref{eqn: C1u}), when}
t &= 0 \\
\intertext{then,}
u &= 0 
\end{align*}
Also, when
\begin{align*}
t \rightarrow \infty \\
\intertext{then,}
u \rightarrow \infty \\
\end{align*}
This indicates that the limits of the integral remain unchanged, provided $s>0$ as stated previously.
\begin{align}
\Leftrightarrow F(s) &= \int_{0}^{\infty} e^{-u} \textcolor{red}{3}\ln \bigg(\frac{u}{s}\bigg) \textcolor{red}{\frac{1}{s}} \,du  \nonumber \\
&= \textcolor{red}{\frac{3}{s}} \int_{0}^{\infty} e^{-u} \ln \bigg(\frac{u}{s}\bigg) \,du  \nonumber \\
&= \frac{3}{s} \int_{0}^{\infty} e^{-u} (\ln u - \ln s) \,du  \nonumber \\
&= \frac{3}{s} \bigg( \int_{0}^{\infty} e^{-u} \ln u \,du - \ln s  \int_{0}^{\infty} e^{-u} \,du \bigg) \nonumber \\
&= \frac{3}{s} \bigg(\bigg( \int_{0}^{\infty} e^{-u} \ln u \,du \bigg) + (\ln s) e^{-u} \bigg|_0^\infty \bigg) \nonumber \\
&= \frac{3}{s} \bigg(\bigg( \int_{0}^{\infty} e^{-u} \ln u \,du \bigg) + (\ln s) (0-1) \bigg) \nonumber \\
F(s)&= \frac{3}{s} \bigg(\bigg( \int_{0}^{\infty} e^{-u} \ln u \,du \bigg) - \ln s \bigg) \label{eqn: C1halfway}
\end{align}
Now, we need to introduce the Euler-Mascheroni constant which is defined as follows:
\begin{align}
\gamma = -\int_{0}^{\infty} e^{-x} \ln x \,dx \label{eqn: Euler-Mascheroni}
\end{align}
Application of (\ref{eqn: Euler-Mascheroni}) to (\ref{eqn: C1halfway}):
\begin{align}
F(s) &= \frac{3}{s} \bigg( -\gamma - \ln s \bigg) \nonumber \\
F(s) &= \frac{-3(\gamma + \ln s)}{s}, s>0 \label{eqn: C1final}
\end{align}
which is the Laplace transform of $f(t)$.


\subsection{Probelm C2}\label{sec:c2}

The following function is defined in the Problem Description:

\begin{equation}
    f(t) = |\cos(\omega t)|, t \geq 0, \omega > 0
\end{equation}

This is a periodic function with period $T = \pi$. Then, 

\begin{align}
f_0(t) &= |cos(\omega t)|(1 - H_\pi(t)) \\
&= \cos(\omega t)(1 - H_\pi(t)) \\
&= \cos(\omega t) - \cos(\omega t)H_\pi(t)
\end{align}

The Laplace transform of $f_0$ can be determined by applying Theorem 4.14 (Shifted (delayed) functions) which can be found on page 95 of the textbook.

\begin{align}
(\lap f_0)(s) &= \lap\{\cos(\omega t)\} - \lap\{\cos(\omega t)H_\pi(t)\} \\
&= \lap\{\cos(\omega t)\} - \lap\{\cos(\omega t - \pi)H_\pi(t)\}
\end{align}

Using the following property of Laplace transforms:
\begin{equation}
    \lap\left\{e(t - t_0)H_{t_0}(t)\right\} = e^{-t_0 s}\lap{w(t)}
\end{equation}

therefore

\begin{align}
     (\lap f_0)(s) &= \frac{s}{s^2 + \omega^2} - e^{-\pi s}\lap\{\cos(\omega t)\} \\
     &= \frac{s}{s^2 + \omega^2} - e^{-\pi s}\frac{s}{s^2 + \omega^2} \\
     &= \frac{s - e^{-\pi s}s}{s^2 + \omega^2} \\
     &= \frac{s(1 - e^{-\pi s})}{s^2 + \omega^2}
\end{align}

From Theorem 4.24 (Laplace transform of periodic function, v2) found on page 105 of the textbook

\begin{align}
(\lap|{\cos(\omega t)|})(s) &= \frac{1}{1 - e^{-\pi s}}\frac{s(1 - e^{-\pi s})}{s^2 + \omega^2} \\
&= \frac{s}{s^2 + \omega^2}
\end{align}


\subsection{Problem C3}\label{sec:c3}

It is given that

\begin{equation}\label{eq:pntwise_lap}
    \lim_{n \to \infty} \big(\lap\left\{f_n\right\}\big) = \lap\{f\}
\end{equation}
\vspace{1pt}

Let $f_n(t)$ be the sequence of time-domain functions, where

\begin{equation}
     f_n(t) = \frac{t}{n}
\end{equation}
\vspace{1pt}

then

\begin{subequations}
     \begin{align}
         \lim_{n \to \infty} \left(\frac{t}{n}\right) &\to 0 \\
         \therefore f(t) &= 0
     \end{align}
\end{subequations}
\vspace{1pt}

therefore, $f_n(t)$ converges pointwise to $f(t)$ for all $t\geq0$

\begin{subequations}
     \begin{align}
         F_n(s) &= \lap\left\{f_n\right\} \\
         &= \lap\left\{\frac{t}{n}\right\} \\
         &= \frac{1}{n} \lap\left\{t\right\} \\
         &= \frac{1}{n} \cdot \frac{1}{s^2} \\
         &= \frac{1}{ns^2}
    \end{align}
\end{subequations}
\vspace{1pt}

therefore, all $f_n(t)$ have a Laplace transform $F_n(s)$

\begin{equation}\label{eq:F}
     F(s) = \lap\{f\} = \lap\{0\} = 0
\end{equation}
\vspace{1pt}

therefore, f(t) has a Laplace transform F(s) \\

It can also be seen that this follows the rule in equation \ref{eq:pntwise_lap}, since

\begin{equation}
     \lim_{n \to \infty} \left(\frac{1}{ns^2}\right) = \frac{1}{s^2} \lim_{n \to \infty} \left(\frac{1}{n}\right) \to 0
\end{equation}
\vspace{1pt}

A counterexample to equation \ref{eq:pntwise_lap} is

\begin{equation}\label{eq:fn}
     f_n(t) = e^t\big(H_{n^2 - n}(t) - H_{n^2 + n}(t)\big)
\end{equation}
\vspace{1pt}

$f(t)$ can be calculated by applying the limit chain rule

\begin{equation}
     f(t) = \lim_{n \to \infty} \bigg(e^t\big(H_{n^2 - n}(t) - H_{n^2 + n}(t)\big)\bigg)
\end{equation}
\vspace{1pt}

let $u(n) = n^2 - n$, $w(n) = n^2 + n$, and $g(t) = e^t\big(H_u(t) - H_w(t)\big)$

\begin{subequations}
     \begin{align}
         \lim_{n \to \infty} (n^2 - n) &\to \infty \\
         \text{and } \lim_{n \to \infty} (n^2 + n) &\to \infty 
     \end{align}  
\end{subequations}
\vspace{1pt}

therefore

\begin{equation}
     f(t) = \lim_{n \to \infty} \bigg(e^t\big(H_{\infty}(t) - H_{\infty}(t)\big)\bigg)
\end{equation}
\vspace{1pt}

since $H_a(t)$ is defined as

\begin{equation}
    H_a(t) = 
     \begin{cases}
         0 & \text{if } t < a \\
         1 & \text{if } t > a
     \end{cases}
\end{equation}
\vspace{1pt}

therefore

\begin{equation}
     H_{\infty}(t) \equiv 0
\end{equation}

then

\begin{subequations}
     \begin{align}
          f(t) &= \lim_{n \to \infty} (e^t \cdot 0) \\
          \therefore f(t) &= 0 \\
          \text{and } F(s) &= 0 \label{eq:F}
     \end{align}
\end{subequations}
\vspace{1pt}

Now $F_n(s)$ must be found

\begin{equation}
     F_n(s) = \lap\left\{e^t H_{n^2 - n}(t) - e^t H_{n^2 + n}(t)\right\}
\end{equation}
\vspace{1pt}

Due to the linearity rule of Laplace transfroms this can be written as

\begin{equation}
     F_n(s) = \lap\left\{e^t H_{n^2 - n}(t)\right\} - \lap\left\{e^t H_{n^2 + n}(t)\right\}
\end{equation}
\vspace{1pt}

The transform rule can now be applied. It states that if $\lap\left\{f(t)\right\} = F(s)$, then $\lap\left\{e^{at} f(t)\right\} = F(s - a)$. So $F_n(s)$ becomes

\begin{equation}
     F_n(s - 1) = \frac{e^{-(n^2 - n)(s-1)}}{s-1} - \frac{e^{-(n^2 + n)(s-1)}}{s-1}
\end{equation}
\vspace{1pt}

therefore

\begin{equation}
     \lim_{n \to \infty} \big(\lap\{f_n\}\big) = \lim_{n \to \infty} \left(\frac{e^{-(n^2 - n)(s-1)}}{s-1} - \frac{e^{-(n^2 + n)(s-1)}}{s-1}\right)
\end{equation}
\vspace{1pt}

Then, by the linearity rule

\begin{equation}
     \lim_{n \to \infty} \big(\lap\{f_n\}\big) = \lim_{n \to \infty} \left(\frac{e^{-(n^2 - n)(s-1)}}{s-1}\right) - \lim_{n \to \infty}\left(\frac{e^{-(n^2 + n)(s-1)}}{s-1}\right)
\end{equation}
\vspace{1pt}

Evaluate the first limit with the limit chain rule. Let $u(n) = n^2 - n$ and $G(s - 1) = \frac{e^{-u(s - 1)}}{s - 1}$

\begin{equation}
     \lim_{n \to \infty} (n^2 - n) \to \infty
\end{equation}
\vspace{1pt}

Then it can be seen that

\begin{equation}
     \lim_{n \to \infty} = \frac{e^{-\infty(s - 1)}}{s - 1} \to \text{Undefined}
\end{equation}
\vspace{1pt}

since $\infty(s - 1)$ is undefined. Therefore

\begin{equation}\label{eq:lim_Fn}
     \lim_{n \to \infty} \big(\lap\{f_n\}\big) = \lim_{n \to \infty} \left(\frac{e^{-(n^2 - n)(s-1)}}{s-1} - \frac{e^{-(n^2 + n)(s-1)}}{s-1}\right) \to \text{Undefined}
\end{equation}
\vspace{1pt}

Equations \ref{eq:F} and \ref{eq:lim_Fn} show that \ref{eq:pntwise_lap} is not true for the example in \ref{eq:fn}. Therefore it can be concluded that for a sequence of time-domain functions $f_n$, equation \ref{eq:pntwise_lap} is only true if $\lim_{n \to \infty} (\lap\{f_n\})$ converges.

\end{document}




